\documentclass{article}

\title{Neural Networks: An Introduction}
\author{Your Name}


\maketitle

\section{Introduction}

Neural networks, also known as artificial neural networks (ANNs), are computational models inspired by the structure and functioning of the human brain. They are widely used in various fields, including machine learning and artificial intelligence. This document provides an introduction to neural networks and their key concepts.

\section{Basic Structure}

A neural network consists of interconnected nodes called \textit{neurons} or \textit{units}. These neurons are organized in layers. Typically, a neural network has three types of layers:

\begin{itemize}
  \item \textbf{Input Layer}: This layer receives the raw data as input. Each neuron in the input layer corresponds to a feature or attribute of the input data.
  
  \item \textbf{Hidden Layers}: These layers are intermediate layers between the input and output layers. They contain multiple neurons and are responsible for learning complex patterns and representations from the input data.
  
  \item \textbf{Output Layer}: This layer produces the final output of the neural network. The number of neurons in the output layer depends on the type of task the neural network is designed for, such as regression or classification.
\end{itemize}

\section{Neuron}

The fundamental building block of a neural network is an artificial neuron. Each neuron takes multiple inputs, applies weights to those inputs, performs a nonlinear transformation, and produces an output. The output of one neuron can serve as an input to other neurons, allowing information to flow through the network.

The inputs to a neuron are multiplied by corresponding weights, and the weighted sum is computed. Additionally, a bias term may be added to the weighted sum. This sum is then passed through an activation function, which introduces nonlinearity to the neuron's output.

\section{Activation Function}

An activation function determines the output of a neuron based on its weighted sum of inputs. It introduces nonlinearity, allowing the neural network to learn complex relationships in the data. Some commonly used activation functions include:

\begin{itemize}
  \item \textbf{Sigmoid}: The sigmoid function maps the weighted sum to a value between 0 and 1, which can be interpreted as a probability.
  
  \item \textbf{ReLU (Rectified Linear Unit)}: The ReLU function returns the input value if it is positive, and zero otherwise. It is a simple and computationally efficient activation function.
  
  \item \textbf{Tanh}: The hyperbolic tangent function maps the weighted sum to a value between -1 and 1, providing a more balanced activation range.
\end{itemize}

\section{Training}

Neural networks learn to make predictions or classify data through a process called \textit{training}. During training, the network adjusts the weights of its neurons to minimize the difference between its predictions and the expected outputs.

One common training algorithm is \textit{backpropagation}, which updates the weights of the neurons by propagating the error backward through the network. This iterative process gradually improves the network's ability to make accurate predictions on new, unseen data.

\section{Applications}

Neural networks have shown remarkable success in various domains, including:

\begin{itemize}
  \item \textbf{Image Recognition}: Neural networks can classify and recognize objects or patterns in images, enabling applications such as facial recognition and object detection.
  
  \item \textbf{Natural Language Processing}: Neural networks are used for language-related tasks, including text
\end{itemize}
